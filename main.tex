\documentclass[14pt,a4paper]{extarticle}
\usepackage[utf8]{inputenc}
\usepackage{amssymb}
\usepackage[polish]{babel}
\usepackage[T1]{fontenc}
\usepackage[left=2cm,right=2cm,top=2.5cm,bottom=3cm]{geometry}

\usepackage{tabularx}
\usepackage{amsmath}
\usepackage{amsfonts}
\usepackage{graphicx}
\usepackage{layout}
\usepackage{multirow}
\usepackage{multicol}
%%% PREAMBUŁA %%%
\graphicspath{{galeria/}}

\AddToHook{shipout/firstpage}{%
   \put (-1in,-27.2cm){\includegraphics[width=\paperwidth,height=\paperheight]{Strona tytułowa.png}}%
}

\setcounter{tocdepth}{2}
\setcounter{secnumdepth}{1}

%%% /PREAMBUŁA %%%



\begin{document}

%%% POCZĄTEK %%%
\pagenumbering{gobble}

\begin{titlepage} % STRONA TYTUŁOWA
\newpage
\
\end{titlepage}

\tableofcontents % SPIS TREŚCI

\cleardoublepage
\pagenumbering{arabic}
%%% /POCZĄTEK %%%



%%% GŁÓWNA TREŚĆ %%%

\section{Symbole i notacja}

\begin{multicols}{2}
\renewcommand{\arraystretch}{1.2}
\setlength{\arrayrulewidth}{0.5mm}

\subsection{Litery greckie}

\begin{tabular}{*{3}{|c}|}
\hline
Nazwa & Mała litera & Duża litera \\
\hline
Alfa & $\alpha $ & $A $ \\ 
Beta & $\beta $ & $B $ \\
Gamma & $\gamma $ & $\Gamma $ \\
Delta & $\delta $ & $\Delta $ \\
Epsilon & $\varepsilon $ & $E $ \\
Dzeta & $\zeta $ & $Z $ \\
Eta & $\eta $ & $H $ \\
Theta & $\theta$, $\vartheta $ & $\Theta $ \\
Jotta & $\iota $ & $I $ \\
Kappa & $\kappa $ & $K $ \\
Lambda & $\lambda $ & $\Lambda $ \\
My & $\mu $ & $M $ \\
Ny & $\nu $ & $N $ \\
Ksi & $\xi $ & $\Xi $ \\
Omikron & $o $ & $O $ \\
Pi & $\pi $ & $\Pi $ \\
Rho & $\rho$, $\varrho$ & $P $ \\
Sigma & $\sigma$, $\varsigma$ & $\Sigma $ \\
Tau & $\tau $ & $T $ \\
Ipsylon & $\upsilon $ & $\Upsilon $ \\
Phi & $\phi$, $\varphi $ & $\Phi $ \\
Chi & $\chi $ & $X $ \\
Psi & $\psi $ & $\Psi $ \\
Omega & $\omega $ & $\Omega $ \\
\hline
\end{tabular}
\\\\\\

\subsection{Zbiory}

\begin{tabular}{*{2}{|c}|}
\hline
Symbol & Znaczenie \\
\hline
$\varnothing $ & Zbiór pusty \\
$A\cup B$ & Suma zbiorów \\
$A\cap B$ & Część wspólna zbiorów \\
$A\setminus B$ & Różnica zbiorów \\
$A\times B$ & Iloczyn kartezjański \\
$\overline{A}$, $A^\prime$ & Dopełnienie zbioru \\
$A\subset B$ & Podzbiór zbioru \\
$A\not\subset B$ & Nie jest podzbiorem zbioru \\
$x\in A$ & Należy do zbioru \\
$x\not\in A$ & Nie należy do zbioru \\
$\vert A\vert $, $\overline{\overline{A}}$ & Liczebność zbioru \\
\hline
\end{tabular}

\subsection{Logika}

\begin{tabular}{*{2}{|c}|}
\hline
Symbol & Znaczenie \\
\hline
$\land$ & I (iloczyn logiczny)\\
$\lor$ & Lub (suma logiczna) \\
$A\Leftrightarrow B$ & Równowartość logiczna \\
$A\Rightarrow B$ & Konsekwencja logiczna \\
$\lnot A$ & Negacja logiczna \\
$A\therefore B$ & Dlatego \\
$A\because B$ & Ponieważ \\
$\forall \;x$, $\underset{x}{\bigwedge}$ & Dla każdego $x$ \\
$\exists \;x$, $\underset{x}{\bigvee}$ & Istnieje $x$ \\
$\exists! \;x$, $\underset{x}{\dot\bigvee}$ & Istnieje dokładnie jeden $x$ \\
\hline
\end{tabular}

\end{multicols}
\newpage
\renewcommand{\arraystretch}{1.2}
\setlength{\arrayrulewidth}{0.5mm}

\subsection{Zbiory liczbowe}

\begin{tabular}{*{4}{|c}|}
\hline
Nazwa & Symbol & Nazwa & Symbol \\
\hline
Naturalne & $\mathbb{N} = \{ 0, 1, 2, 3,\dots\}$ & Wymierne & $\mathbb{Q}$, $\!\mathbb{W} \!= \!\{\frac{p}{q}\!:\!p, q \!\in \!\mathbb{Z} \!\wedge \!q \!\neq \!0 \}$ \\
Naturalne dod. & $\mathbb{N}_{+} = \mathbb{N}\setminus\{0\}$ & Niewymierne & $\mathbb{R}\setminus\mathbb{Q}$, $\mathbb{NW}$ \\
Całkowite & $\mathbb{Z}$, $\mathbb{C} = \{-1, 0, 1,\dots\}$ & Rzeczywiste & $\mathbb{R}$ \\
\hline
\end{tabular}

\subsection{Operacje arytmetyczne}

\begin{tabular}{*{4}{|c}|}
\hline
Symbol & Znaczenie & Symbol & Znaczenie \\
\hline
$a+b$ & Dodawanie & $a < b$ & Mniejsze od \\
$a-b$ & Odejmowanie & $a > b$ & Większe od \\
$a\cdot b$, $a\times b$ & Mnożenie & $a \leq b$ & Mniejsze bądź równe od \\
$a/b$, $\frac{a}{b}$ & Dzielenie & $a \geq b$ & Większe bądź równe od \\
$x^{n}$ & Potęgowanie & $a \approx b$ & Aproksymacja \\
$\sqrt{x}$ & Pierwiastek kwadratowy & $x\%$ & Procent \\
$\sqrt[n]{x}$ & Pierwiaster $n$-tego stopnia & $x$\textperthousand & Promil \\
$\log_{a}x$ & Logarytm o podstawie $a$ & $\vert x\vert$ & Wartość bezwzględna \\
$\log x$ & Logarytm dziesiętny & $\lceil x\rceil$ & Sufit \\
$\ln x$ & Logarytm naturalny & $\lfloor x\rfloor$ & Podłoga \\
$a = b$ & Znak równości & $\{x\}$ & Mantysa (część ułamkowa) \\
$a \neq b$ & Nierówność & $x$ mod $a$ & Dzielenie całkowite (modulo) \\
\hline
\end{tabular}

\begin{multicols}{2}

\subsection{Stochastyka i statystyka}

\begin{tabular}{*{2}{|c}|}
\hline
Symbol & Znaczenie \\
\hline
$n!$ & Silnia \\
$\binom{n}{k}$ & Kombinacja bez powtórzeń \\
$\Omega$ & Przestrzeń probabilistyczna \\
$P(A)$ & Prawdopodobieństwo \\
$P(A \mid\! B)$ & Prawdopodobieństwo warunkowe \\
$\sigma^{2}$ & Wariancja \\
$\sigma$ & Odchylenie standardowe \\
$\bar{x}$ & Średnia arytmetyczna \\
\hline
\end{tabular}
\\\\\\

\subsection*{\hspace{2cm}Geometria}
\addcontentsline{toc}{subsection}{Geometria}

\hskip2.0cm
\begin{tabular}{*{2}{|c}|}
\hline
Symbol & Znaczenie \\
\hline
$\vert AB\vert$ & Odcinek \\
$\overset{\longrightarrow}{AB}$ & Wektor \\
$\angle{}$, $\measuredangle{}$, $\sphericalangle{}$ & Kąt \\
$\triangle ABC$ & Trójkąt \\
$\square ABCD$ & Czworokąt \\
$k \parallel l$ & Proste równoległe \\
$k \perp l$ & Proste prostopadłe \\
$\sim$ & Figury podobne \\
$\equiv$ & Figury przystające \\
\hline
\end{tabular}

\end{multicols}

\newpage
\renewcommand{\arraystretch}{1.2}
\setlength{\arrayrulewidth}{0.5mm}

\section{Prawa działań}
\subsection{Wartość bezwzględna}
\textbf{Wartość bezwzględna (moduł liczby)} - operacja, która zwraca nienegatywną wartość. Zdefiniowana jest następującym równaniem:
\begin{equation*}
\lvert x \rvert = \left\{
   \begin{array}{ll}
      x, & x \geq 0 \\
      -x, & x < 0
   \end{array}
\right., x \in \mathbb{R}
\end{equation*}
Dla $a, b \in \mathbb{R}$ prawdziwe są następujące zależności:
\begin{itemize}
   \item Nienegatywność: $\lvert a\rvert \geq 0$,
   \item Określoność dodatnia: $\lvert a \rvert = 0 \Leftrightarrow a = 0$,
   \item Multiplikatywność: $\lvert ab\rvert = \lvert a\rvert\lvert b\rvert$,
   \item Podaddytywność: $\lvert a + b\rvert \leq \lvert a\rvert + \lvert b \rvert,\;\; \lvert a - b\rvert \geq \lvert a\rvert - \lvert b \rvert$,
   \item Idempotencja: $\lvert\lvert a \rvert\rvert = \lvert a \rvert$,
   \item Parzystość: $\lvert -a\rvert = \lvert a\rvert$,
   \item Zasada identyczności przedmiotów nierozróżnialnych: $\lvert a - b \rvert = 0 \Leftrightarrow a = b$,
   \item Zachowanie dzielenia: $\left\lvert\frac{\displaystyle a}{\displaystyle b}\right\rvert = \frac{\displaystyle \lvert a\rvert}{\displaystyle \lvert b \rvert} \Leftrightarrow b \neq 0$,
\end{itemize}
Dodatkowo:
\begin{equation*}
\begin{array}{ccc}
   \lvert a\rvert = \sqrt{a}^{2}, &\hspace{1cm} \lvert a \rvert \leq b \Leftrightarrow -b \leq a \leq b, &\hspace{1cm} \lvert a \rvert \geq b \Leftrightarrow a \leq -b \lor a \geq b \\
\end{array}
\end{equation*}

\subsection{Potęgi i pierwiastki}
\textbf{Potęgowanie (podniesienie do $n$-tej potęgi)} - operacja dwuargumentowa, która jest zdefiniowana jako iloczyn $a, a \in \mathbb{R}\setminus\{0\}$(podstawa) $n, n \in \mathbb{N}^{+}$(wykładnik) razy:\
$$a^{n} = \underbrace{a\cdot a \cdot\ldots\cdot a}_{n\;\;razy}$$
Szczególne przypadki:
\begin{equation*}
\begin{array}{ccc}
   a^{1} = a, &\hspace{1cm} a^{0} = 1, &\hspace{1cm} 0^{n} = 0 \\ 
\end{array}
\end{equation*}
\renewcommand{\arraycolsep}{1cm}
\renewcommand{\arraystretch}{1.7}
\noindent\textbf{Pierwiastkowanie} - operacja odwrotna do potęgowania, która dla $a,\linebreak 
a=\{x:x\in\mathbb{R}\land x\geq 0\}$ zwraca liczb(ę/y) $b, b\in\mathbb{R}$, która pomnożona $n,\linebreak 
n=\{x:x\in\mathbb{N}\land x\geq 2\}$ razy jest równa $b$:
$$b = \sqrt[n]{a} \Leftrightarrow b^{n} = a$$

\noindent Dla $a,b \in \mathbb{R}, b \neq 0; m,n \in \mathbb{N}, n \neq 0 $ prawdziwe są następujące zależności:
\begin{equation*}
\begin{array}{ll}
   a^{-n} = \frac{\displaystyle 1}{\displaystyle a^{n}} & \sqrt[n]{a} = a^{\frac{1}{n}} \\
   a^{-\frac{m}{n}} = \frac{\displaystyle 1}{ \sqrt[n]{\displaystyle a^{m}}} & \sqrt[n]{a^{m}} = a^{\frac{m}{n}}\\
   (a\cdot b)^{n} = a^{n}\cdot b^{n} & \sqrt[n]{a\cdot b} = \sqrt[n]{a} \cdot \sqrt[n]{b} \\
   \left(\frac{\displaystyle a}{\displaystyle b}\right)^{n} = \frac{\displaystyle a^{n}}{\displaystyle b^{n}} & \sqrt[n]{\frac{\displaystyle a}{\displaystyle b}} = \frac{\displaystyle \sqrt[n]{a}}{\displaystyle \sqrt[n]{b}} \\
   (a^{n})^{m} = a^{n\cdot m} & \sqrt[m]{\sqrt[n]{a}} = \sqrt[m\cdot n]{a}\\
   a^{n}\cdot a^{m} = a^{n+m} & \frac{\displaystyle a^{n}}{\displaystyle a^{m}} = a^{n - m} \\
\end{array}
\end{equation*}
\vspace{-0.45cm}
\begin{equation*}
\begin{array}{c}
\sqrt[n]{a^{m}} = (\sqrt[n]{a})^{m}
\end{array}
\end{equation*}


\end{document}