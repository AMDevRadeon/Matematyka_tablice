\documentclass[14pt,a4paper]{extarticle}
\usepackage[utf8]{inputenc}
\usepackage{amssymb}
\usepackage[polish]{babel}
\usepackage[T1]{fontenc}
\usepackage[left=2cm,right=2cm,top=2.5cm,bottom=3cm]{geometry}

\usepackage{tabularx}
\usepackage{amsmath}
\usepackage{amsfonts}
\usepackage{graphicx}
\usepackage{layout}
\usepackage{multirow}
\usepackage{multicol}
%%% PREAMBUŁA %%%
\graphicspath{{galeria/}}

\AddToHook{shipout/firstpage}{%
   \put (-1in,-27.2cm){\includegraphics[width=\paperwidth,height=\paperheight]{Strona tytułowa.png}}%
}

\setcounter{tocdepth}{2}
\setcounter{secnumdepth}{1}

%%% /PREAMBUŁA %%%



\begin{document}

%%% POCZĄTEK %%%
\pagenumbering{gobble}

\begin{titlepage} % STRONA TYTUŁOWA
\newpage
\
\end{titlepage}

\tableofcontents % SPIS TREŚCI

\cleardoublepage
\pagenumbering{arabic}
%%% /POCZĄTEK %%%



%%% GŁÓWNA TREŚĆ %%%

\section{Symbole i notacja}

\begin{multicols}{2}
\renewcommand{\arraystretch}{1.2}
\setlength{\arrayrulewidth}{0.5mm}

\subsection{Litery greckie}

\begin{tabular}{*{3}{|c}|}
\hline
Nazwa & Mała litera & Duża litera \\
\hline
Alfa & $\alpha $ & $A $ \\ 
Beta & $\beta $ & $B $ \\
Gamma & $\gamma $ & $\Gamma $ \\
Delta & $\delta $ & $\Delta $ \\
Epsilon & $\varepsilon $ & $E $ \\
Dzeta & $\zeta $ & $Z $ \\
Eta & $\eta $ & $H $ \\
Theta & $\theta$, $\vartheta $ & $\Theta $ \\
Jotta & $\iota $ & $I $ \\
Kappa & $\kappa $ & $K $ \\
Lambda & $\lambda $ & $\Lambda $ \\
My & $\mu $ & $M $ \\
Ny & $\nu $ & $N $ \\
Ksi & $\xi $ & $\Xi $ \\
Omikron & $o $ & $O $ \\
Pi & $\pi $ & $\Pi $ \\
Rho & $\rho$, $\varrho$ & $P $ \\
Sigma & $\sigma$, $\varsigma$ & $\Sigma $ \\
Tau & $\tau $ & $T $ \\
Ipsylon & $\upsilon $ & $\Upsilon $ \\
Phi & $\phi$, $\varphi $ & $\Phi $ \\
Chi & $\chi $ & $X $ \\
Psi & $\psi $ & $\Psi $ \\
Omega & $\omega $ & $\Omega $ \\
\hline
\end{tabular}
\\\\\\

\subsection{Zbiory}

\begin{tabular}{*{2}{|c}|}
\hline
Symbol & Znaczenie \\
\hline
$\varnothing $ & Zbiór pusty \\
$A\cup B$ & Suma zbiorów \\
$A\cap B$ & Część wspólna zbiorów \\
$A\setminus B$ & Różnica zbiorów \\
$A\times B$ & Iloczyn kartezjański \\
$\overline{A}$, $A^\prime$ & Dopełnienie zbioru \\
$A\subset B$ & Podzbiór zbioru \\
$A\not\subset B$ & Nie jest podzbiorem zbioru \\
$x\in A$ & Należy do zbioru \\
$x\not\in A$ & Nie należy do zbioru \\
$\vert A\vert $, $\overline{\overline{A}}$ & Liczebność zbioru \\
\hline
\end{tabular}

\subsection{Logika}

\begin{tabular}{*{2}{|c}|}
\hline
Symbol & Znaczenie \\
\hline
$\land$ & I (iloczyn logiczny)\\
$\lor$ & Lub (suma logiczna) \\
$A\Leftrightarrow B$ & Równowartość logiczna \\
$A\Rightarrow B$ & Konsekwencja logiczna \\
$\lnot A$ & Negacja logiczna \\
$A\therefore B$ & Dlatego \\
$A\because B$ & Ponieważ \\
$\forall \;x$, $\underset{x}{\bigwedge}$ & Dla każdego $x$ \\
$\exists \;x$, $\underset{x}{\bigvee}$ & Istnieje $x$ \\
$\exists! \;x$, $\underset{x}{\dot\bigvee}$ & Istnieje dokładnie jeden $x$ \\
\hline
\end{tabular}

\end{multicols}
\newpage
\renewcommand{\arraystretch}{1.2}
\setlength{\arrayrulewidth}{0.5mm}

\subsection{Zbiory liczbowe}

\begin{tabular}{*{4}{|c}|}
\hline
Nazwa & Symbol & Nazwa & Symbol \\
\hline
Naturalne & $\mathbb{N} = \{ 0, 1, 2, 3,\dots\}$ & Wymierne & $\mathbb{Q}$, $\!\mathbb{W} \!= \!\{\frac{p}{q}\!:\!p, q \!\in \!\mathbb{Z} \!\wedge \!q \!\neq \!0 \}$ \\
Naturalne dod. & $\mathbb{N}_{+} = \mathbb{N}\setminus\{0\}$ & Niewymierne & $\mathbb{R}\setminus\mathbb{Q}$, $\mathbb{NW}$ \\
Całkowite & $\mathbb{Z}$, $\mathbb{C} = \{-1, 0, 1,\dots\}$ & Rzeczywiste & $\mathbb{R}$ \\
\hline
\end{tabular}

\subsection{Operacje arytmetyczne}

\begin{tabular}{*{4}{|c}|}
\hline
Symbol & Znaczenie & Symbol & Znaczenie \\
\hline
$a+b$ & Dodawanie & $a < b$ & Mniejsze od \\
$a-b$ & Odejmowanie & $a > b$ & Większe od \\
$a\cdot b$, $a\times b$ & Mnożenie & $a \leq b$ & Mniejsze bądź równe od \\
$a/b$, $\frac{a}{b}$ & Dzielenie & $a \geq b$ & Większe bądź równe od \\
$x^{n}$ & Potęgowanie & $a \approx b$ & Aproksymacja \\
$\sqrt{x}$ & Pierwiastek kwadratowy & $x\%$ & Procent \\
$\sqrt[n]{x}$ & Pierwiaster $n$-tego stopnia & $x$\textperthousand & Promil \\
$\log_{a}x$ & Logarytm o podstawie $a$ & $\vert x\vert$ & Wartość bezwzględna \\
$\log x$ & Logarytm dziesiętny & $\lceil x\rceil$ & Sufit \\
$\ln x$ & Logarytm naturalny & $\lfloor x\rfloor$ & Podłoga \\
$a = b$ & Znak równości & $\{x\}$ & Mantysa (część ułamkowa) \\
$a \neq b$ & Nierówność & $x$ mod $a$ & Dzielenie całkowite (modulo) \\
\hline
\end{tabular}

\begin{multicols}{2}

\subsection{Stochastyka i statystyka}

\begin{tabular}{*{2}{|c}|}
\hline
Symbol & Znaczenie \\
\hline
$n!$ & Silnia \\
$\binom{n}{k}$ & Kombinacja bez powtórzeń \\
$\Omega$ & Przestrzeń probabilistyczna \\
$P(A)$ & Prawdopodobieństwo \\
$P(A \mid\! B)$ & Prawdopodobieństwo warunkowe \\
$\sigma^{2}$ & Wariancja \\
$\sigma$ & Odchylenie standardowe \\
$\bar{x}$ & Średnia arytmetyczna \\
\hline
\end{tabular}
\\\\\\

\subsection*{\hspace{2cm}Geometria}
\addcontentsline{toc}{subsection}{Geometria}

\hskip2.0cm
\begin{tabular}{*{2}{|c}|}
\hline
Symbol & Znaczenie \\
\hline
$\vert AB\vert$ & Odcinek \\
$\overset{\longrightarrow}{AB}$ & Wektor \\
$\angle{}$, $\measuredangle{}$, $\sphericalangle{}$ & Kąt \\
$\triangle ABC$ & Trójkąt \\
$\square ABCD$ & Czworokąt \\
$k \parallel l$ & Proste równoległe \\
$k \perp l$ & Proste prostopadłe \\
$\sim$ & Figury podobne \\
$\equiv$ & Figury przystające \\
\hline
\end{tabular}

\end{multicols}

\newpage
\renewcommand{\arraystretch}{1.2}
\setlength{\arrayrulewidth}{0.5mm}

\section{Prawa działań}
\subsection{Wartość bezwzględna}
\textbf{Wartość bezwzględna (moduł liczby)} - operacja, która zwraca nienegatywną wartość. Zdefiniowana jest następującym równaniem:
\begin{equation*}
\lvert x \rvert = \left\{
   \begin{array}{ll}
      x, & x \geq 0 \\
      -x, & x < 0
   \end{array}
\right., x \in \mathbb{R}
\end{equation*}
Dla $a, b \in \mathbb{R}$ prawdziwe są następujące zależności:
\begin{itemize}
   \item Nienegatywność: $\lvert a\rvert \geq 0$,
   \item Określoność dodatnia: $\lvert a \rvert = 0 \Leftrightarrow a = 0$,
   \item Multiplikatywność: $\lvert ab\rvert = \lvert a\rvert\lvert b\rvert$,
   \item Podaddytywność: $\lvert a + b\rvert \leq \lvert a\rvert + \lvert b \rvert,\;\; \lvert a - b\rvert \geq \lvert a\rvert - \lvert b \rvert$,
   \item Idempotencja: $\lvert\lvert a \rvert\rvert = \lvert a \rvert$,
   \item Parzystość: $\lvert -a\rvert = \lvert a\rvert$,
   \item Zasada identyczności przedmiotów nierozróżnialnych: $\lvert a - b \rvert = 0 \Leftrightarrow a = b$,
   \item Zachowanie dzielenia: $\left\lvert\frac{\displaystyle a}{\displaystyle b}\right\rvert = \frac{\displaystyle \lvert a\rvert}{\displaystyle \lvert b \rvert} \Leftrightarrow b \neq 0$,
\end{itemize}
Dodatkowo:
\begin{equation*}
\begin{array}{ccc}
   \lvert a\rvert = \sqrt{a}^{2}, &\hspace{1cm} \lvert a \rvert \leq b \Leftrightarrow -b \leq a \leq b, &\hspace{1cm} \lvert a \rvert \geq b \Leftrightarrow a \leq -b \lor a \geq b \\
\end{array}
\end{equation*}

\subsection{Potęgi, pierwiastki i logarytmy}
\textbf{Potęgowanie (podniesienie do $n$-tej potęgi)} - operacja dwuargumentowa, która jest zdefiniowana jako iloczyn $a, a \in \mathbb{R}\setminus\{0\}$(podstawa) $n, n \in \mathbb{N}_{+}$(wykładnik) razy:\
$$a^{n} = \underbrace{a\cdot a \cdot\ldots\cdot a}_{n\;\;razy}$$
Szczególne przypadki:
\begin{equation*}
\begin{array}{ccc}
   a^{1} = a, &\hspace{1cm} a^{0} = 1, &\hspace{1cm} 0^{n} = 0 \\ 
\end{array}
\end{equation*}

\renewcommand{\arraycolsep}{0cm}
\renewcommand{\arraystretch}{2}


\noindent\textbf{Pierwiastkowanie} - operacja odwrotna do potęgowania, która dla $a$ przyjmuje wartość $b$ taką, że pomnożona $n$ razy jest równa $b$:

\begin{equation*}
\begin{array}{ccc}
    & b = \sqrt[n]{a} \Leftrightarrow b^{n} = a, & \\
   a=\{x:x\in\mathbb{R}\land x\geq 0\}, & b\in\mathbb{R}, & n=\{x:x\in\mathbb{N}\land x\geq 1\} \\
\end{array}
\end{equation*}
\noindent Dla $a,b \in \mathbb{R}, b \neq 0; m,n \in \mathbb{N}, n \neq 0 $ prawdziwe są następujące zależności:

\renewcommand{\arraycolsep}{1cm}
\renewcommand{\arraystretch}{2}

\begin{equation*}
\begin{array}{ll}
   a^{-n} = \frac{\displaystyle 1}{\displaystyle a^{n}} & \sqrt[n]{a} = a^{\frac{1}{n}} \\
   a^{-\frac{m}{n}} = \frac{\displaystyle 1}{ \sqrt[n]{\displaystyle a^{m}}} & \sqrt[n]{a^{m}} = a^{\frac{m}{n}}\\
   (a\cdot b)^{n} = a^{n}\cdot b^{n} & \sqrt[n]{a\cdot b} = \sqrt[n]{a} \cdot \sqrt[n]{b} \\
   \left(\frac{\displaystyle a}{\displaystyle b}\right)^{n} = \frac{\displaystyle a^{n}}{\displaystyle b^{n}} & \sqrt[n]{\frac{\displaystyle a}{\displaystyle b}} = \frac{\displaystyle \sqrt[n]{a}}{\displaystyle \sqrt[n]{b}} \\
   (a^{n})^{m} = a^{n\cdot m} & \sqrt[m]{\sqrt[n]{a}} = \sqrt[m\cdot n]{a}\\
   \\
   a^{n}\cdot a^{m} = a^{n+m} & \frac{\displaystyle a^{n}}{\displaystyle a^{m}} = a^{n - m} \\
   \sqrt[n]{a^{m}} = (\sqrt[n]{a})^{m} &  \sqrt{a}^{2} = \lvert a\rvert \\
\end{array}
\end{equation*}
\\

\renewcommand{\arraycolsep}{0.5cm}

\noindent\textbf{Logarytm} - operacja odwrotna do potęgowania, która dla podstawy $a$ oraz argumentu $b$ przyjmuje wartość $n$ taką, że $a$ podniesione do potęgi $n$ jest\linebreak równe $b$:
$$n = \log_{\;a}b \Leftrightarrow a^{n} = b$$
\begin{equation*}
\begin{array}{ccc}
   a=\{x:x\in\mathbb{R}\land x>0\land x\neq 1\}, & b \in \mathbb{R}_{+}, & n \in \mathbb{R} \\
\end{array}
\end{equation*}

\noindent Szczególne przypadki:
\begin{equation*}
\begin{array}{ccc}
   \log_{\;a}0 - \text{niezdefiniowany}, & \log_{\;a}1 = 0, & \log_{\;a}a = 1 \\
\end{array}
\end{equation*}

\renewcommand{\arraycolsep}{0.8cm}
\newpage
\noindent Dla $a, b =\{x:x\in\mathbb{R}\land x>0\land x\neq 1\}; x, y \in \mathbb{R}_{+}$ prawdziwe są następujące zależności:

\begin{itemize}
   \item Prawo iloczynu: $\log_{\;a}(x\cdot y) = \log_{\;a}x + \log_{\;a}y$
   \item Prawo ilorazu: $\log_{\;a}\left(\frac{\displaystyle x}{\displaystyle y}\right) = \log_{\;a}x - \log_{\;a}y$
   \item Prawo potęgi: $\log_{\;a}(x^{y}) = y\cdot\log_{\;a}x$
   \item Zamiana podstawy z argumentem: $\log_{\;a}b = \frac{\displaystyle 1}{\displaystyle \log_{\;b}a}$
   \item Zmiana podstawy logarytmu: $\log_{\;a}x = \frac{\displaystyle \log_{\;b}x}{\displaystyle \log_{\;b}a}$
   \item Logarytm potęgi podstawy: $\log_{\;a}(a^{x}) = x$
   \item $a$ do potęgi logarytmu $a$ z $x$: $a^{\log_{\;a}x} = x$
\end{itemize}

\noindent\subsection{Wzory skróconego mnożenia}
Dla $x, y \in \mathbb{R}$ prawdziwe są następujące zależności:
\begin{itemize}
   \item Kwadrat sumy: $(x + y)^{2} = x^{2} + 2xy + y^{2}$
   \item Kwadrat różnicy: $(x - y)^{2} = x^{2} - 2xy + y^{2}$
   \item Różnica kwadratów: $x^{2} - y^{2} = (x - y)(x + y)$
   \\
   \item Sześcian sumy: $(x + y)^{3} = x^{3} + 3x^{2}y + 3xy^{2} + y^{3}$
   \item Sześcian różnicy: $(x - y)^{3} = x^{3} - 3x^{2}y + 3xy^{2} - y^{3}$
   \item Różnica sześcianów: $x^{3} - y^{3} = (x - y)(x^{2} + xy + y^{2})$
   \item Suma sześcianów: $x^{3} + y^{3} = (x + y)(x^{2} - xy + y^{2})$
\end{itemize}

\newpage
\noindent Za pomocą \textbf{trójkąta Pascala} można wyznaczyć współczynniki arugmentów dla sumy i różnicy podniesionej do potęgi dowolnego $n, n \in \mathbb{N}$:\\
\begin{tabular}{>{$n=}l<{$\hspace{4pt}}*{17}{c}}
   0 &&&&&&&&&&&1&&&&&&\\
   1 &&&&&&&&&&1&&1&&&&&\\
   2 &&&&&&&&&1&&2&&1&&&&\\
   3 &&&&&&&&1&&3&&3&&1&&&\\
   4 &&&&&&&1&&4&&6&&4&&1&&\\
   5 &&&&&&1&&5&&10&&10&&5&&1&\\
   6 &&&&&1&&6&&15&&20&&15&&6&&1\\
\end{tabular}
\hfill\break
\textit{Przykład:}
$\;\;\;(x - y)^{5} = x^{5} - 5x^{4}y + 10x^{3}y^{2} - 10x^{2}y^{3} + 5xy^{4} - y^{5}$

\noindent\subsection{Średnie}

\begin{itemize}
   \item Arytmetyczna: $S_{a} = $\scalebox{1.1}{$\frac{\displaystyle x_{1} + x_{2} + x_{3} + \ldots + x_{n}}{\textstyle n}$}$;\; x_{1}, x_{2}, x_{3}, \ldots, x_{n} \in \mathbb{R}$
   \item Geometryczna: $S_{g} = $\scalebox{1.2}{$\sqrt[n]{x_{1} \cdot x_{2} \cdot x_{3} \cdot \ldots \cdot x_{n}}$}$;\; x_{1}, x_{2}, x_{3}, \ldots, x_{n} \in \mathbb{R}_{+}$
   \item Kwadratowa: $S_{k} = \sqrt{\frac{\displaystyle x_{1}^{2} + x_{2}^{2} + x_{3}^{2} + \ldots + x_{n}^{2}}{\displaystyle n}};\; x_{1}, x_{2}, x_{3}, \ldots, x_{n} \in \mathbb{R}$
   \item Harmoniczna: $S_{h} = $\scalebox{1.4}{$\frac{\textstyle n}{\frac{1}{x_{1}}+\frac{1}{x_{2}}+\frac{1}{x_{3}}+\ldots+\frac{1}{x_{n}}}$}$;\; x_{1}, x_{2}, x_{3}, \ldots, x_{n} \in \mathbb{R} \!\setminus\!\{0\} $
\end{itemize}
\textbf{Nierówność Cauchy'ego między średnimi} - średnie wyznaczone dla tego samego układu liczb dodatnich układają się w charakterystyczną nierówność:
$$S_{k} \geq S_{a} \geq S_{g} \geq S_{h}$$

\newpage
\noindent\subsection{Błędy przybliżenia}
Dla $r$ oznaczającego wartość dokładną i $p$ oznaczającego wartość przybliżoną zdefiniowane są błędy przybliżenia:\hfill\break
\noindent\textbf{Błąd bezwzględny przybliżenia} - wartość bezwzględna różnicy między wartością dokładną a przybliżoną, wyrażona wzorem: $\vert r - p\vert$.\hfill\break
\textbf{Błąd względny przybliżenia} - iloraz błędu bezwzględnego i wartości bezwzględnej rzeczywistej wielkości, wyrażona: $\frac{\displaystyle\vert r - p\vert}{\displaystyle\vert r\vert}$.\hfill\break
\textbf{Błąd procentowy przybliżenia} - wartość procentowa błędu względnego: \hfill\break$\frac{\displaystyle\vert r - p\vert}{\displaystyle\vert r\vert} \cdot 100\%$.\\
\newpage

\section{Wielomiany}
\subsection{Informacje i twierdzenia}
\textbf{Wielomianem stopnia \textit{n}, \textit{n} $\in \mathbb{N}_{+}$ zmiennej rzeczywistej x} nazywamy wyrażenie:
$$W(x) = a_{n}x^{n}+a_{n-1}x^{n-1}+a_{n-2}x^{n-2}+\ldots+a_{2}x^{2}+a_{1}x+a_{0},$$
gdzie $a_{0}, a_{1}, a_{2}, \ldots, a_{n} \in \mathbb{R} \land a_{n} \neq 0$. Liczby $a_{0}, a_{1}, a_{2}, \ldots, a_{n}$ to współczynniki wielomianu.
Wielomian stopnia zerowego to każda liczba rzeczywista różna od zera. Wielomian zerowy to liczba równa zeru; nie ma określonego stopnia.\\
\smallskip

\noindent\textbf{Wielomian W(\textit{x}) jest podzielny przez wielomian P(\textit{x})} rózny od wielomianu zerowego wtedy i tylko wtedy,
gdy istnieje taki wielomian Q(\textit{x}), dla którego prawdziwa jest równość:
$$W(x) = P(x) \cdot Q{x}$$
Wówczas Q(\textit{x}) nazywany jest ilorazem wielomianu W(\textit{x}) przez P(\textit{x}), a P(\textit{x}) 
jest dzielnikiem wielomianu W(\textit{x}).\\

\smallskip
\noindent\textbf{Pierwiastek wielomianu W(\textit{x})} to liczba rzeczywista $a$, dla której $W(a) = 0$.\\

\smallskip
\noindent\textbf{Pierwiastek \textit{k}-krotny wielomianu W(\textit{x})}, gdzie $k \in \mathbb{N}_{+}$ to liczba $a$ taka, że W(\textit{x})
jest podzielny przez $(x-a)^{k}$ i nie jest podzielny przez $(x-a)^{k+1}$. Liczba $k$ jest nazywana krotnością pierwiastka.\\

\bigskip

\noindent\textbf{Twierdzenie o dzieleniu (rozkładzie) wielomianu}\\
Jeśli W(\textit{x}) oraz P(\textit{x}) są wielomianami i P(\textit{x}) nie jest wielomianem zerowym, 
to istnieją dwa wielomiany Q(\textit{x}) oraz R(\textit{x}) takie, że prawdziwa jest równość:
$$W(x) = P(x)\cdot Q(\textit{x})+R(x),$$
gdzie R(\textit{x}) jest wielomianem zerowym lub wielomianem o stopniu mniejszym od stopnia wielomianu P(\textit{x}).\\

\smallskip
\noindent\textbf{Twierdzenie o reszcie wielomianu}\\
Reszta z dzielenia wielomianu W(\textit{x}) przez dwumian $(x - a)$ jest równa W(\textit{a}).\\

\smallskip
\newpage
\noindent\textbf{Twierdzenie o wymiernych pierwiastkach wielomianu o współczynnikach całkowitych}\\
Jeżeli wielomian W(\textit{x}) ma pierwiastek wymierny, który można zapisać w postaci nieskracalnego ułamka $\frac{p}{q}$, gdzie $p, q \in \mathbb{C} \land q \neq 0$, to 
$p$ jest dzielnikiem wyrazu wolnego $a_{0}$, natomiast $q$ - dzielnikiem współczynnika $a_{n}$ przy najwyższej potędze zmiennej.\\

\smallskip
\noindent\textbf{Twierdzenie Bezouta}\\
Liczba $a$ jest pierwiastkiem wielomianu W(\textit{x}) wtedy i tylko wtedy, gdy wielomian W(\textit{x}) jest podzielny przez dwumian $(x - a)$.\\

\smallskip
\noindent\textbf{Twierdzenie o liczbie pierwiastków wielomianu}\\
Każdy wielomian stopnia $n$ ma co najwyżej $n$ pierwiastków.\\

\smallskip
\noindent\textbf{Twierdzenie o rozkładzie wielomianu na czynniki}\\
Każdy wielomian stopnia co najmniej trzeciego można rozłożyć na czynniki sponia co najwyżej drugiego.
Rozkład w taki sposób jest jednoznaczny (z dokładnością co do kolejności czynników i stałej).\\

\subsection{Funkcja liniowa}
\noindent\textbf{Wzór ogólny:}
$$f(x) = ax + b;\;\; a, b \in \mathbb{R}$$
Liczba $a$ nazywana jest współczynnikiem kierunkowym, $b$ wyrazem wolnym. Dziedziną i zbiorem wartości
funkcji liniowej są liczby rzeczywiste.\hfill\break
\vspace{1pt}

\noindent Funkcja liniowa przecina oś OY w punkcie $(0, b)$, a oś OX w $x_{0} = -\frac{b}{a}$. Wykres funkcji liniowej 
jest nachylony do osi OX pod kątem $\alpha$ takim, że $\text{tg}\alpha = a$.\\

\noindent Funkcja liniowa jest:
\begin{itemize}
   \item nieograniczona,
   \item nieokresowa,
   \item monotoniczna ($a > 0 \text{ - rosnąca}, a < 0 \text{ - malejąca}, a = 0 \text{ - stała}$),
   \item różnowartościowa (gdy $a \neq 0$),
   \item ciągła,
   \item różniczkowalna.
\end{itemize}

\subsection{Funkcja kwadratowa}
\noindent\textbf{Wzór w postaci ogólnej:}
$$f(x) = ax^{2} + bx + c;\;\; a, b, c \in \mathbb{R}, a \neq 0$$
\noindent\textbf{Wzór w postaci kanonicznej:}
$$f(x) = a(x - p)^{2} + q;\;\; p = -\frac{b}{2a},\;\; q = -\frac{\Delta}{4a}, \;\; \Delta = b^{2} -4ac$$
\noindent\textbf{Wzór w postaci iloczynowej:}
$$f(x) = a(x - x_{1})(x - x_{2}) \Leftrightarrow \Delta \geq 0; \;\; x_{1} = \frac{-b + \sqrt{\Delta}}{2a},\;\; x_{2} = \frac{-b - \sqrt{\Delta}}{2a}$$

\noindent\textbf{Delta, wyróżnik wielomianu stopnia drugiego} - Liczba opisująca relację między współczynnikami funkcji 
kwadratowej a jej miejscami zerowymi. Wyznaczona wzorem $\Delta_{2} = b^{2} - 4ac$.
\begin{itemize}
   \item $\Delta > 0$ - Dwa różne pierwiastki rzeczywiste: $x_{1}, x_{2} = \frac{-b \pm\sqrt{\Delta}}{2a}$,
   \item $\Delta = 0$ - Jeden pierwiastek rzeczywisty podwójny: $x_{0} = -\frac{b}{2a}$,
   \item $\Delta < 0$ - Brak pierwiastków rzeczywistych: ($x_{1}, x_{2} \in \mathbb{C}$\;(zespolonych)).
\end{itemize}

\noindent Wykresem funkcji kwadratowej jest parabola, której ramiona skierowane są do góry gdy $a > 0$ a do dołu,
gdy $a < 0$.\hfill\break


\noindent\textbf{Wierzchołek funkcji kwadratowej} znajduje się w punkcie $\left(-\frac{b}{2a}, -\frac{\Delta}{4a}\right)$.
Alternatywnie, współrzędną \textbf{x} wierzchołka można wyliczyć średnią arytmetyczną miejsc zerowych: $p = \frac{x_{1} + x_{2}}{2}$, a 
współrzędną \textbf{y} wartością funkcji w punkcie $p$: $q = f(p)$.\hfill\break

\noindent\textbf{Wzory Viète'a} - wzory wiążące współczynniki wielomianu (stopnia drugiego) z jego pierwiastkami:\hfill\break
\begin{equation*}
   \left\{
      \begin{array}{ll}
         \!\!\!\!\!\!x_{1} + x_{2} = -\frac{\displaystyle b}{\displaystyle a}\\
         \!\!\!\!\!\!x_{1}\cdot x_{2} = \frac{\displaystyle c}{\displaystyle a}\\
      \end{array}
   \right.
   \end{equation*}

\noindent Dziedziną funkcji kwadratowej jest zbiór liczb rzeczywistych. Zbiorem wartości jest przedział
$\langle q, +\infty) \text{ dla } a > 0$, $(-\infty, q\rangle \text{ dla } a < 0$.\hfill\break

\noindent Funkcja kwadratowa jest:
\begin{itemize}
   \item nieokresowa,
   \item monotoniczna w przedziałach:
   $$\text{malejąca w }(-\infty, p\rangle, \text{ rosnąca w } \langle p, +\infty),\;\; a > 0$$
   $$\text{rosnąca w }(-\infty, p\rangle, \text{ malejąca w } \langle p, +\infty),\;\; a < 0$$
   \item ciągła,
   \item różniczkowalna,
   \item ściśle wypukła dla $a > 0$, ściśle wklęsła dla $a < 0$.

\end{itemize}


% \begin{equation*}
% \begin{array}{cc}
%     &  \\
% \end{array}
% \end{equation*}

% \vspace{-0.7cm}

% \begin{equation*}
% \begin{array}{ccc}
%     &  &   \\
% \end{array}
% \end{equation*}

% \vspace{-0.3cm}

% \begin{equation*}
% \begin{array}{cc}
%     &  \\
% \end{array}
% \end{equation*}


\end{document}