\documentclass[14pt,a4paper]{extarticle}
\usepackage[utf8]{inputenc}
\usepackage{amssymb}
\usepackage[polish]{babel}
\usepackage[T1]{fontenc}
\usepackage[left=2cm,right=2cm,top=2.5cm,bottom=3cm]{geometry}

\usepackage{amsmath}
\usepackage{amsfonts}
\usepackage{graphicx}
\usepackage{layout}
\usepackage{multirow}
\usepackage{multicol}
%%% PREAMBUŁA %%%
\graphicspath{{galeria/}}

\AddToHook{shipout/firstpage}{%
   \put (-1in,-27.2cm){\includegraphics[width=\paperwidth,height=\paperheight]{Strona tytułowa.png}}%
}

\setcounter{tocdepth}{2}
\setcounter{secnumdepth}{1}

%%% /PREAMBUŁA %%%



\begin{document}

%%% POCZĄTEK %%%
\pagenumbering{gobble}

\begin{titlepage} % STRONA TYTUŁOWA
\newpage
\
\end{titlepage}

\tableofcontents % SPIS TREŚCI

\cleardoublepage
\pagenumbering{arabic}
%%% /POCZĄTEK %%%



%%% GŁÓWNA TREŚĆ %%%

\section{Symbole i notacja}

\begin{multicols*}{2}
\renewcommand{\arraystretch}{1.2}
\setlength{\arrayrulewidth}{0.5mm}

\subsection{Litery greckie}

\begin{tabular}{*{3}{|c}|}
\hline
Nazwa & Mała litera & Duża litera \\
\hline
Alfa & $\alpha $ & $A $ \\ 
Beta & $\beta $ & $B $ \\
Gamma & $\gamma $ & $\Gamma $ \\
Delta & $\delta $ & $\Delta $ \\
Epsilon & $\varepsilon $ & $E $ \\
Dzeta & $\zeta $ & $Z $ \\
Eta & $\eta $ & $H $ \\
Theta & $\theta$, $\vartheta $ & $\Theta $ \\
Jotta & $\iota $ & $I $ \\
Kappa & $\kappa $ & $K $ \\
Lambda & $\lambda $ & $\Lambda $ \\
My & $\mu $ & $M $ \\
Ny & $\nu $ & $N $ \\
Ksi & $\xi $ & $\Xi $ \\
Omikron & $o $ & $O $ \\
Pi & $\pi $ & $\Pi $ \\
Rho & $\rho$, $\varrho$ & $P $ \\
Sigma & $\sigma$, $\varsigma$ & $\Sigma $ \\
Tau & $\tau $ & $T $ \\
Ipsylon & $\upsilon $ & $\Upsilon $ \\
Phi & $\phi$, $\varphi $ & $\Phi $ \\
Chi & $\chi $ & $X $ \\
Psi & $\psi $ & $\Psi $ \\
Omega & $\omega $ & $\Omega $ \\
\hline
\end{tabular}

\subsection{Zbiory liczbowe}

\begin{tabular}{*{2}{|c}|}
\hline
Nazwa & Symbol \\
\hline
Naturalne & $\mathbb{N} = \{ 0, 1, 2, 3,\dots\}$ \\
Naturalne dod. & $\mathbb{N}_{+} = \mathbb{N}\setminus\{0\}$ \\
\hline
\end{tabular}
\begin{tabular}{*{2}{|c}|}
\hline
Całkowite & $\mathbb{Z}$, $\mathbb{C} = \{-1, 0, 1, 2,\dots\}$ \\
Wymierne & $\mathbb{Q}$, $\mathbb{W} = \{\frac{p}{q}:p, q \in \mathbb{Z} \wedge q \neq 0 \}$ \\
Niewymierne & $\mathbb{R}\setminus\mathbb{Q}$, $\mathbb{NW}$ \\
Rzeczywiste & $\mathbb{R}$ \\
\hline
\end{tabular}

\subsection{Zbiory}

\begin{tabular}{*{2}{|c}|}
\hline
Symbol & Znaczenie \\
\hline
$\varnothing $ & Zbiór pusty \\
$A\cup B$ & Suma zbiorów \\
$A\cap B$ & Część wspólna zbiorów \\
$A\setminus B$ & Różnica zbiorów \\
$A\times B$ & Iloczyn kartezjański \\
$\overline{A}$, $A^\prime$ & Dopełnienie zbioru \\
$A\subset B$ & Podzbiór zbioru \\
$A\not\subset B$ & Nie jest podzbiorem zbioru \\
$x\in A$ & Należy do zbioru \\
$x\not\in A$ & Nie należy do zbioru \\
$\vert A\vert $, $\overline{\overline{A}}$ & Liczebność (kardynalność) zbioru \\
\hline
\end{tabular}

\subsection{Logika}

\begin{tabular}{*{2}{|c}|}
\hline
Symbol & Znaczenie \\
\hline
$\land$ & Koniunkcja (iloczyn logiczny) \\
$\lor$ & Alternatywa (suma logiczna) \\
$A\Leftrightarrow B$ & Równowartość logiczna \\
$A\Rightarrow B$ & Konsekwencja logiczna \\
$\lnot A$ & Negacja logiczna \\
\hline
\end{tabular}

\end{multicols*}
\newpage
\renewcommand{\arraystretch}{1.2}
\setlength{\arrayrulewidth}{0.5mm}

\subsection{Operacje arytmetyczne}

\begin{tabular}{*{4}{|c}|}
\hline
Symbol & Znaczenie & Symbol & Znaczenie \\
\hline
$a+b$ & Dodawanie & $a < b$ & Mniejsze od \\
$a-b$ & Odejmowanie & $a > b$ & Większe od \\
$a\cdot b$, $a\times b$ & Mnożenie & $a \leq b$ & Mniejsze bądź równe od \\
$a/b$, $\frac{a}{b}$ & Dzielenie & $a \geq b$ & Większe bądź równe od \\
$x^{n}$ & Potęgowanie & $a \approx b$ & Aproksymacja \\
$\sqrt{x}$ & Pierwiastek kwadratowy & $x\%$ & Procent \\
$\sqrt[n]{x}$ & Pierwiaster $n$-tego stopnia & $x$\textperthousand & Promil \\
$\log_{a}x$ & Logarytm o podstawie $a$ & $\vert x\vert$ & Wartość bezwzględna \\
$\log x$ & Logarytm dziesiętny & $\lceil x\rceil$ & Podłoga \\
$\ln x$ & Logarytm naturalny & $\lfloor x\rfloor$ & Sufit \\
$a = b$ & Znak równości & $\{x\}$ & Mantysa (część ułamkowa) \\
$a \neq b$ & Nierówność & $x$ mod $a$ & Dzielenie całkowite (modulo) \\
\hline
\end{tabular}

\begin{multicols*}{2}
\end{multicols*}

\end{document}